
\chapter{Reflection and Conclusion}

As a learning experience, overall the project was successful and gratifying in that the end product is a piece of well tested, functional and useful software. In this Chapter, we briefly discuss what went well, and what could be improved upon concerning the software's development.

\section{Management of work}

The management strategy involving the use of Trello boards was overall successful. Work was done consistently throughout the allotted timeframe. Milestones were adjusted as work continued, but no significant changes were made. In hindsight, these milestones should have been slightly tighter throughout the year to allow time to address feedback from the second focus group session.

Furthermore, as discussed in Chapter 2, it was decided to complete the littlegit-core first, followed by the server and GUI. The rationale being that this would make the development of the latter two easier. However, this resulted in a small amount of code written in the littlegit-core that is never used by the other components. By employing the concept of \emph{thin, end to end slices of functionality} as is preferred by Agile methodologies we could have avoided this.


\section{Project goals}

All the goals discussed in Chapter 1 were achieved, the desired functionality is in place, and feedback from users suggests it is useable and fit for purpose.  In short, we are content with the overall results of the project and consider it a success.


\section{What went wrong?}

Of course, there were areas in which things could have been improved. The most significant change made if this project were to be restarted is the technology stack. Kotlin as a language indeed was a good choice. However, the use of the Tornado FX framework proved to be a mistake. The problem lies in the fact it is built upon JavaFX which unfortunately the developer community seems to have rejected.

Support for Java FX online is minimal, making development more difficult and more importantly making the software tough to distribute and install. Installing the software on machines in schools is highly unlikely to be feasible, which is a major disadvantage to software aimed at teenagers. In hindsight, despite the massive disadvantages of JavaScript as a language, use of the Electron framework which we disregarded in Chapter 3, may have been the better choice to address these issues.

\section{Personal Achievements}

I learnt a great deal in completing this project. Learning the Kotlin language including its functional aspects was certainly a valuable experience. This includes using it in writing a library and learning to deploy it using Maven.

Though University did teach the basics of using Git, this project required digging much deeper into low-level Git commands and architecture. As a result, I believe myself considerably more knowledgable about Git, and its quirks than I was beforehand. 

Scalability is also an issue I have never before had to consider in writing software at University. It was very new to me to try to write code designed to be run simultaneously on multiple machines (and multiple threads) and am very happy with the results! On a related note, security and authentication (especially using OAuth 2.0) is something I have never had to consider during my University career. Through this project, I learned about securing endpoints with token-based authentication, and of course about using more advanced features of SSH to secure the Git repositories themselves.

Lastly, I gained valuable experience in running focus group sessions, something I had never done before to obtain feedback from likely users. These sessions were incredibly valuable and indeed a form of obtaining feedback I will use in future!