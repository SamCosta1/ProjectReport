Git commands are loosely grouped into two categories (though the distinction is not a hard one). The first is git-porcelain. Porcelain commands are those with which most Git users are likely to have interacted.

A good example is \texttt{git-status}, which gives a summary of the current state of the user's local Git repository. These are commands designed for use by human users; often, their outputs are tweaked between Git versions to aid human readability. This makes them unsuited for machine use; it is impossible to reliably parse output which is slightly different in every version of Git.

Plumbing commands, on the other hand, are generally lower level commands to the porcelain and are often used internally by the porcelain itself. Their outputs are considered stable and designed for interaction with other software rather than humans. While making them a far better choice for the littlegit-core to interact with, the difficulty is in the fact they are by nature lower level commands. For example, to check out a branch with git-porcelain is a straightforward one-line operation, but this conceals the invocation of multiple plumbing commands. The littlegit-core endeavours to use plumbing commands wherever possible to ensure compatibility with future Git versions.
